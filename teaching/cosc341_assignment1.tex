\documentclass{article}

\usepackage[hidelinks]{hyperref}
\usepackage{fancyhdr}
\usepackage{amsthm}

\setlength{\textwidth}{15cm}
\setlength{\textheight}{21cm}
\setlength{\topmargin}{0cm}
\setlength{\headsep}{0.5cm}
\setlength{\headheight}{0.5cm}
\setlength{\oddsidemargin}{0.25pt}
\setlength{\evensidemargin}{0.25pt}
\setlength{\parindent}{0pt}
\setlength{\parskip}{5pt}

\makeatletter
\newfont{\footsc}{cmcsc10 at 8truept}
\newfont{\footbf}{cmbx10 at 8truept}
\newfont{\footrm}{cmr10 at 10truept}

\renewcommand{\footrulewidth}{0.4pt}

\pagestyle{fancy}
\fancyhead[LO, LE]{{\sc COSC} 341: Assignment 1}

\begin{document}


\noindent {\bf Due Date: Thursday 28 March, 4pm}

Please submit a {\bf PDF} by email to: \href{mailto:teaching@biods.org}{teaching@biods.org} with subject ``341 Assignment 1''. Make sure that your name and ID number are on the PDF called: YourName\_assignment1.pdf

What you hand in should be all your own work.

Please note the following
\begin{itemize}
\item There are 22 marks in this assignment but the \textbf{maximum} you can get is \textbf{10}.
\item Marks outside the set $\{0, 10\}$ are very uncommon for question 3.
\item Unless we receive a correct solution to question 3, the solution will not be made available within this course.
\end{itemize}

\section*{Questions}
\begin{enumerate}
\item
\begin{enumerate}
\item Assume that a gene is a word over the alphabet $\{A, T, C, G\}$.
	Assume also that a protein is unambiguously defined by a finite set of genes and different sets of genes define different proteins.
	Is the set of all proteins finite or infinite?
	If infinite, is the set countable or uncountable?
	Justify your answers formally (with proofs). (\textbf{3 marks})
\item Consider $F$, the set of all functions $f: \{0,1,2\}^* \to \{0,1,2\}^*$ mapping from ternary numbers to ternary numbers.
	Is $F$ countable or uncountable?
	Justify your answer formally. (\textbf{3 marks})
\end{enumerate}

\item Design an automaton for each of the following languages over the alphabet $\Sigma = \{a, b, c\}$.
	If your automaton is non-deterministic, provide a DFA for the same language.
	Explain why your DFA recognises the same language.
	If you decide to use the automaton determinisation algorithm, list all steps of the algorithm you make. (\textbf{2 marks each}):
	\begin{enumerate}
	  \item The set of strings in which every $ab$ is immediately followed by a $ac$.
	  \item The set of strings that contain $aba$ or $bab$ (or both) as a substring.
	  \item The set of strings that begin with $a$, contain exactly two $b$'s and end with $cc$.
	\end{enumerate}

\item Prove that the Pumping Lemma is not a sufficient condition for a language to be automatic, that is, the Pumping Lemma is not an ``if and only if'' statement. (\textbf{10 marks})
\end{enumerate}
\end{document}
